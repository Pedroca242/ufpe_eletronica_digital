\chapter{Conclusão}\label{cap_6_Conclusao}

Pode-se concluir que vários objetivos foram concluídos. O banco de registradores foi implementado. Os quatro push-buttons, seus LED's e suas devidas sonoridas foram implementadas. O contador MOD2 também foi implementado.

%Como limitação, o contador MOD10 autoral não é decrescente como foi requisitado. Ele nesse projeto é crescente, mas com pequenas alterações ele poderia ser crescente.

Houve dificuldades na relação teoria versus prática, pois os push-buttons requeriram um debounce para confirmar que eles estavam, de fato, sendo precionados. O contador MOD10 apresentava erros na FPGA quando havia sua mudança para ser decrescente.

A realização do relatório foi muito importante para o grupo. Isso porque, foi pela realização dele, que foi possível ter discussões sobre como melhorar o projeto e sobre o que cada arquivo do quartus fazia. Também foi possível, para cada integrante do grupo, compreender melhor todas as partes do projeto, já que a elaboração do relatório incentivou o diálogo entre todos os integrantes do grupo e, consequentemente, levou a um melhor trabalho na equipe.



