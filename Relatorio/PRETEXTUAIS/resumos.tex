% ---
% RESUMOS
% ---
\begin{comment}
% resumo em português
\setlength{\absparsep}{18pt} % ajusta o espaçamento dos parágrafos do resumo
\begin{resumo}

%==========================================
\hspace{1.1cm}
“Malware” é uma junção dos termos “malicioso” e “software”. O malware tem como principal objetivo acessar um dispositivo alheio sem permissão explícita de seu proprietário. 
Dentre os malwares, as Ameaças Persistentes Avançadas, do inglês \textit{Advanced Persistent Threat} (APT), ganharam muito espaço no tópico de roubo de dados e comportamento destrutivo para softwares, principalmente quando se trata de organizações federais e indústrias privadas de grande porte, devido à complexidade e eficiência desse tipo de ataque. Os malwares do tipo APT são direcionado para um alvo pré-definido sendo sempre bem orquestrado com grande precisão e controle, aproveitando os recursos de reconhecimento e vulnerabilidades avançadas.
%----------------------------------------
 O presente trabalho propõe uma análise crítica acerca do desempenho dos principais antivírus comerciais atuais quanto à detecção de malwares do tipo APT.
 Em acréscimo, são replicados antivírus do estado da arte dotados de distintas metodologias de detecção baseadas no princípio de redes neurais. 
 Como contribuição principal, um antivírus autoral é criado por meio de uma máquina de aprendizagem extrema e com kernels de processamento pseudo-morfológicos. O referido antivírus autoral tem o intuito de apresentar uma alternativa criativa, eficaz e rápida no controle desse tipo de malware. 
Por último, são apresentados os resultados da aplicação e sua comparação com os demais antivírus do estado da arte.
%----------------------------------------
O antivírus autoral alcança um desempenho médio de 93,62\% na
distinção entre aplicativos benignos e APT acompanhado de um tempo treinamento de 0,55 segundos, em média.
Espera-se que o antivírus inteligente autoral atue de forma preventiva e impeça que os malwares do tipo APT
causem prejuízos às instituições privadas e autarquias públicas. 
%----------------------------------------

\textbf{Palavras-chave}: Antivírus, Detecção de malwares, Ameaças persistentes Avançadas, Máquina de aprendizagem extrema.
\end{resumo}

% resumo em inglês
\begin{resumo}[ABSTRACT]
 \begin{otherlanguage*}{english}
\hspace{1.1cm} "Malware" is a joining of the terms "malicious" and "software." Malware is primarily intended to access another's device without the owner's explicit permission. 
Among malware, Advanced Persistent Threats (APT) have gained much ground on the topic of data theft and destructive behavior for software, especially when it comes to federal organizations and large private industries, due to the complexity and efficiency of this type of attack. APT-type malware is directed at a predefined target and is always well-orchestrated with great precision and control, taking advantage of advanced vulnerability recognition capabilities.
%----------------------------------------
 This paper proposes a critical analysis of the performance of the current main commercial antivirus products in detecting APT malware.
 In addition, state of the art antivirus programs with different detection methodologies based on the neural network principle are replicated. 
 As a main contribution, an authoral antivirus is created by means of an extreme learning machine with pseudo-morphological processing kernels. This authorial antivirus is intended to present a creative, effective and fast alternative in the control of this type of malware. 
Finally, the results of the application and its comparison with other state of the art antiviruses are presented.
%----------------------------------------
The author antivirus reaches an average performance of 93.62\% in
distinction between benign and APT applications accompanied by a training time of 0.55 seconds on average.
The intelligent authoring antivirus is expected to act preventively and stop APT malware from malware from harming private institutions and public autarchies. 

   \vspace{\onelineskip}
 
   \noindent 
   \textbf{Keywords}: Antivirus, Malware detection, Advanced persistent threats, Extreme learning machine.
 \end{otherlanguage*}
\end{resumo}

%% resumo em francês 
%\begin{resumo}[Résumé]
% \begin{otherlanguage*}{french}
%    Il s'agit d'un résumé en français.
% 
%   \textbf{Mots-clés}: latex. abntex. publication de textes.
% \end{otherlanguage*}
%\end{resumo}
%
%% resumo em espanhol
%\begin{resumo}[Resumen]
% \begin{otherlanguage*}{spanish}
%   Este es el resumen en español.
%  
%   \textbf{Palabras clave}: latex. abntex. publicación de textos.
% \end{otherlanguage*}
%\end{resumo}
% ---
\end{comment}