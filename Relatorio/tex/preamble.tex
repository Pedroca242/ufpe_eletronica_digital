% -----------------------------
\usepackage[utf8x]{inputenc} % coding
 \usepackage[portuguese,brazilian]{babel}
\usepackage{amsmath}
\usepackage{amsthm}
\usepackage{amsfonts}
\usepackage{amssymb}
\usepackage{mathrsfs}

\usepackage{nicefrac}

\usepackage{amsthm}
\interdisplaylinepenalty=2500 % To restore IEEEtran’s ability to automatically do page breaks within multiline equations
\usepackage{bm} % for nice bold symbols

\newtheorem{theorem}{Teorema}
\newtheorem{lemma}{Lema}

%\theoremstyle{definition}
\newtheorem{definition}{Defini\c c\~ao}
\newtheorem{exemplo}{Exemplo}

\usepackage{multirow}
\usepackage{qrcode} % To create QR Codes!
\usepackage{wrapfig} % To wrap text around figures (or QR Codes)

\usepackage{blindtext} % Generate Lorem Ipsum

\usepackage[width=0.00cm, left=3.00cm, right=2.00cm, top=2.00cm, bottom=2.00cm]{geometry} % conFigura as margens
% \usepackage{multicol}
\setlength{\columnsep}{0.8cm}

\usepackage{epigraph}
\setlength{\epigraphwidth}{.6\textwidth}

% FONT CONFIG -----------------
\usepackage{palatino} 
% \usepackage[scaled=1]{helvet}
% \usepackage{courier}

\usepackage[font={small}, labelfont={bf}, margin=1cm]{caption}

\usepackage{indentfirst}
\usepackage{graphicx}
%\usepackage[font={small}, labelfont={bf}, margin=1cm]{subcaption}
\usepackage{subfig}

% \usepackage{steinmetz}
%\usepackage{parskip}
% \setlength{\parindent}{0.7cm}

% \usepackage{pdfpages} % Allow adding pdf files to the document

\usepackage{enumitem} % For custom spacing in lists environments
\setlist{nolistsep}

\usepackage{booktabs}

\usepackage{longtable}
%\usepackage[usenames, dvipsnames]{color}

\usepackage{mdframed} % For fancy frames

% \usepackage{fancyhdr}
% \DeclareGraphicsExtensions{.pdf,.png,.jpg}

\usepackage{float}
% \usepackage{subcaption}

% =========================================
% New commands ---------------
\newcommand{\qi}{\boldsymbol{i}}
\newcommand{\qj}{\boldsymbol{j}}
\newcommand{\qk}{\boldsymbol{k}}
\newcommand{\qmu}{\boldsymbol{\mu}}
\newcommand{\qnu}{\boldsymbol{\nu}}
\newcommand{\qV}{\boldsymbol{V}}
\newcommand{\cas}{\mathrm{cas}}

% ISOMORPHISM SYMBOL

\makeatletter
\newcommand*{\isomorphism}{%
	\mathrel{%
		\mathpalette\@isomorphism{}%
	}%
}
\newcommand*{\@isomorphism}[2]{%
	% Calculate the amount of moving \sim up as in \simeq
	\sbox0{$#1\simeq$}%
	\sbox2{$#1\sim$}%
	\dimen@=\ht0 %
	\advance\dimen@ by -\ht2 %
	%
	% Compose the two symbols
	\sbox0{%
		\lower1.9\dimen@\hbox{%
			$\m@th#1\relbar\isomorphism@joinrel\relbar$%
		}%
	}%
	\rlap{%
		\hbox to \wd0{%
			\hfill\raise\dimen@\hbox{$\m@th#1\sim$}\hfill
		}%
	}%
	\copy0 %
}
\newcommand*{\isomorphism@joinrel}{%
	\mathrel{%
		\mkern-3.4mu %
		\mkern-1mu %
		\nonscript\mkern1mu %
	}%
}
\makeatother
% =========================================

% % Formatando e criando hiperlinks
\usepackage{hyperref}
\hypersetup{
	colorlinks=true,
	linkcolor=blue,
	filecolor=magenta,      
	urlcolor=blue,
	pdfpagemode=FullScreen,
	citecolor=blue,
	linktoc=all,
}


\usepackage{url} % Permite exibição de sites de forma organizada


% \usepackage[nottoc,numbib]{tocbibind} % Adiciona a bibliografia à ToC



% BIBTEX ---------------------
%\bibliographystyle{plain}
%\usepackage[hyperref=true,
%url=false,
%isbn=false,
%doi=false,
%backref=false,
%style=numeric,
%citestyle=numeric-comp,
%sorting=none,
%block=none,
%maxnames=99]{biblatex}
%\usepackage{csquotes}
%\addbibresource{mybib.bib}
\usepackage{cite}


% LISTINGS CONFIG --------------
\usepackage{listings}
% \definecolor{mygreen}{rgb}{0,0.6,0}
% \definecolor{mygray}{rgb}{0.5,0.5,0.5}
% \definecolor{mymauve}{rgb}{0.58,0,0.82}
\lstset{ %
	backgroundcolor=\color{white},   % choose the background color; you must add \usepackage{color} or \usepackage{xcolor}
	basicstyle=\footnotesize,        % the size of the fonts that are used for the code
	breakatwhitespace=false,         % sets if automatic breaks should only happen at whitespace
	breaklines=true,                 % sets automatic line breaking
	captionpos=b,                    % sets the caption-position to bottom
	commentstyle=\color{OliveGreen},    % comment style
	%   deletekeywords={...},            % if you want to delete keywords from the given language
	escapeinside={\%*}{*)},          % if you want to add LaTeX within your code
	extendedchars=true,              % lets you use non-ASCII characters; for 8-bits encodings only, does not work with UTF-8
	frame=single,	                   % adds a frame around the code
	keepspaces=true,                 % keeps spaces in text, useful for keeping indentation of code (possibly needs columns=flexible)
	keywordstyle=\color{blue},       % keyword style
	language=Python,                 % the language of the code
	%   otherkeywords={*,...},           % if you want to add more keywords to the set
	numbers=left,                    % where to put the line-numbers; possible values are (none, left, right)
	%   numbersep=5pt,                   % how far the line-numbers are from the code
	numberstyle=\color{black}, % the style that is used for the line-numbers
	rulecolor=\color{black},         % if not set, the frame-color may be changed on line-breaks within not-black text (e.g. comments (green here))
	%   showspaces=false,                % show spaces everywhere adding particular underscores; it overrides 'showstringspaces'
	showstringspaces=false,          % underline spaces within strings only
	%   showtabs=false,                  % show tabs within strings adding particular underscores
	stepnumber=5,                    % the step between two line-numbers. If it's 1, each line will be numbered
	stringstyle=\color{magenta},     % string literal style
	tabsize=2,	                   % sets default tabsize to 2 spaces
	%   caption=\lstname                   % show the filename of files included with \lstinputlisting; also try caption instead of title
}

\renewcommand{\lstlistingname}{Código}

% PSTricks -----------------
\usepackage{pst-all}