\chapter{Código main em C}
\label{cap:mainc}

\begin{minted} {c} {

#include <stdio.h>
#include <unistd.h>
#include <fcntl.h>
#include <sys/mman.h>
#include "hwlib.h"
#include "socal/socal.h"
#include "socal/hps.h"
#include "socal/alt_gpio.h"
#include "hps_0.h"

#define REGS_BASE (ALT_LWFPGASLVS_OFST)
#define REGS_SPAN (0x00200000)
#define HW_REGS_MASK ( HW_REGS_SPAN - 1 )
int main()
{
	printf("1\n");

	int fd;
	void *virtual_base;
	int loop_count;
	void *pwm_addr;
	void *ram_base_addr;
//	void *ram_addr;
	int counter = 0;
	
	int duty_cycle1 = 90;
	int duty_cycle2 = 50;
	
	printf("2\n");

	printf("ALT_STM_OFST Value: 0x%08x\n", ALT_STM_OFST);
	printf("ALT_LWFPGASLVS_OFST Value: 0x%08x\n", ALT_LWFPGASLVS_OFST);

	// map the address space for the LED registers into user space so we can interact with them.
	// we'll actually map in the entire CSR span of the HPS since we want to access various registers within that span
	if ((fd = open( "/dev/mem", (O_RDWR | O_SYNC))) == -1) 
	{
		printf("ERROR: could not open \"/dev/mem\"...\n");
		return -1;
	}

	printf("3\n");

	virtual_base = mmap(NULL, REGS_SPAN, PROT_READ | PROT_WRITE, MAP_SHARED, fd, REGS_BASE);

	if (virtual_base == MAP_FAILED) 
	{
		printf("ERROR: mmap() failed...\n");
		close(fd);
		return -1;
	}
	
	printf("4\n");

	pwm_addr = virtual_base + LED_BASE;
	ram_base_addr = virtual_base + ONCHIP_MEMORY2_0_BASE;
	
	// Guardando os valores de duty cycle na RAM
	
	*(uint32_t *)(ram_base_addr) = duty_cycle1;
	
	usleep(1000*100);
	
	*(uint32_t *)(ram_base_addr + 0x4) = duty_cycle2;
	
	// Varia o duty cycle dentro do Loop e manda para o PIO conectado a um LED
	
	while (loop_count < 100) 
	{	
		printf("5\n");
		
		//*(uint32_t *)(ram_base_addr) = counter;
		usleep(100*100);
		
		// Set LEDs to counter value
		*(uint32_t *)pwm_addr = counter;
		usleep(1000*100);
		printf("6\n");

		// Wait 0.5s
		usleep(500*1000);
		
		//*(uint32_t *)ram_base_addr = counter;
		// Update the counter so it is never more than 8 bits
		if (counter++ >= 256)
			counter = 0;

		printf("Counter: %i\n", counter);

		// Increment loop counter
		++loop_count;		
	}	
	
	// Clean up our memory mapping and exit
	if (munmap(virtual_base, REGS_SPAN) != 0) 
	{
		printf("ERROR: munmap() failed...\n");
		close(fd);
		return -1;
	}

	close(fd);
	return 0;
}


\end{minted}